%% bare_jrnl.tex
%% V1.3
%% 2007/01/11
%% by Michael Shell
%% see http://www.michaelshell.org/
%% for current contact information.
%%
%% This is a skeleton file demonstrating the use of IEEEtran.cls
%% (requires IEEEtran.cls version 1.7 or later) with an IEEE journal paper.
%%
%% Support sites:
%% http://www.michaelshell.org/tex/ieeetran/
%% http://www.ctan.org/tex-archive/macros/latex/contrib/IEEEtran/
%% and
%% http://www.ieee.org/



% *** Authors should verify (and, if needed, correct) their LaTeX system  ***
% *** with the testflow diagnostic prior to trusting their LaTeX platform ***
% *** with production work. IEEE's font choices can trigger bugs that do  ***
% *** not appear when using other class files.                            ***
% The testflow support page is at:
% http://www.michaelshell.org/tex/testflow/


%%*************************************************************************
%% Legal Notice:
%% This code is offered as-is without any warranty either expressed or
%% implied; without even the implied warranty of MERCHANTABILITY or
%% FITNESS FOR A PARTICULAR PURPOSE! 
%% User assumes all risk.
%% In no event shall IEEE or any contributor to this code be liable for
%% any damages or losses, including, but not limited to, incidental,
%% consequential, or any other damages, resulting from the use or misuse
%% of any information contained here.
%%
%% All comments are the opinions of their respective authors and are not
%% necessarily endorsed by the IEEE.
%%
%% This work is distributed under the LaTeX Project Public License (LPPL)
%% ( http://www.latex-project.org/ ) version 1.3, and may be freely used,
%% distributed and modified. A copy of the LPPL, version 1.3, is included
%% in the base LaTeX documentation of all distributions of LaTeX released
%% 2003/12/01 or later.
%% Retain all contribution notices and credits.
%% ** Modified files should be clearly indicated as such, including  **
%% ** renaming them and changing author support contact information. **
%%
%% File list of work: IEEEtran.cls, IEEEtran_HOWTO.pdf, bare_adv.tex,
%%                    bare_conf.tex, bare_jrnl.tex, bare_jrnl_compsoc.tex
%%*************************************************************************

% Note that the a4paper option is mainly intended so that authors in
% countries using A4 can easily print to A4 and see how their papers will
% look in print - the typesetting of the document will not typically be
% affected with changes in paper size (but the bottom and side margins will).
% Use the testflow package mentioned above to verify correct handling of
% both paper sizes by the user's LaTeX system.
%
% Also note that the "draftcls" or "draftclsnofoot", not "draft", option
% should be used if it is desired that the figures are to be displayed in
% draft mode.
%
\documentclass[journal]{IEEEtran}
\usepackage{graphicx}
\usepackage{minted}
\usepackage{etoolbox}
\AtBeginEnvironment{minted}{\fontsize{10}{10}\selectfont}
		
% Some very useful LaTeX packages include:
% (uncomment the ones you want to load)


% *** MISC UTILITY PACKAGES ***
%
%\usepackage{ifpdf}
% Heiko Oberdiek's ifpdf.sty is very useful if you need conditional
% compilation based on whether the output is pdf or dvi.
% usage:
% \ifpdf
%   % pdf code
% \else
%   % dvi code
% \fi
% The latest version of ifpdf.sty can be obtained from:
% http://www.ctan.org/tex-archive/macros/latex/contrib/oberdiek/
% Also, note that IEEEtran.cls V1.7 and later provides a builtin
% \ifCLASSINFOpdf conditional that works the same way.
% When switching from latex to pdflatex and vice-versa, the compiler may
% have to be run twice to clear warning/error messages.






% *** CITATION PACKAGES ***
%
%\usepackage{cite}
% cite.sty was written by Donald Arseneau
% V1.6 and later of IEEEtran pre-defines the format of the cite.sty package
% \cite{} output to follow that of IEEE. Loading the cite package will
% result in citation numbers being automatically sorted and properly
% "compressed/ranged". e.g., [1], [9], [2], [7], [5], [6] without using
% cite.sty will become [1], [2], [5]--[7], [9] using cite.sty. cite.sty's
% \cite will automatically add leading space, if needed. Use cite.sty's
% noadjust option (cite.sty V3.8 and later) if you want to turn this off.
% cite.sty is already installed on most LaTeX systems. Be sure and use
% version 4.0 (2003-05-27) and later if using hyperref.sty. cite.sty does
% not currently provide for hyperlinked citations.
% The latest version can be obtained at:
% http://www.ctan.org/tex-archive/macros/latex/contrib/cite/
% The documentation is contained in the cite.sty file itself.






% *** GRAPHICS RELATED PACKAGES ***
%
\ifCLASSINFOpdf
  % \usepackage[pdftex]{graphicx}
  % declare the path(s) where your graphic files are
  % \graphicspath{{../pdf/}{../jpeg/}}
  % and their extensions so you won't have to specify these with
  % every instance of \includegraphics
  % \DeclareGraphicsExtensions{.pdf,.jpeg,.png}
\else
  % or other class option (dvipsone, dvipdf, if not using dvips). graphicx
  % will default to the driver specified in the system graphics.cfg if no
  % driver is specified.
  % \usepackage[dvips]{graphicx}
  % declare the path(s) where your graphic files are
  % \graphicspath{{../eps/}}
  % and their extensions so you won't have to specify these with
  % every instance of \includegraphics
  % \DeclareGraphicsExtensions{.eps}
\fi
% graphicx was written by David Carlisle and Sebastian Rahtz. It is
% required if you want graphics, photos, etc. graphicx.sty is already
% installed on most LaTeX systems. The latest version and documentation can
% be obtained at: 
% http://www.ctan.org/tex-archive/macros/latex/required/graphics/
% Another good source of documentation is "Using Imported Graphics in
% LaTeX2e" by Keith Reckdahl which can be found as epslatex.ps or
% epslatex.pdf at: http://www.ctan.org/tex-archive/info/
%
% latex, and pdflatex in dvi mode, support graphics in encapsulated
% postscript (.eps) format. pdflatex in pdf mode supports graphics
% in .pdf, .jpeg, .png and .mps (metapost) formats. Users should ensure
% that all non-photo figures use a vector format (.eps, .pdf, .mps) and
% not a bitmapped formats (.jpeg, .png). IEEE frowns on bitmapped formats
% which can result in "jaggedy"/blurry rendering of lines and letters as
% well as large increases in file sizes.
%
% You can find documentation about the pdfTeX application at:
% http://www.tug.org/applications/pdftex





% *** MATH PACKAGES ***
%
\usepackage[cmex10]{amsmath}
% A popular package from the American Mathematical Society that provides
% many useful and powerful commands for dealing with mathematics. If using
% it, be sure to load this package with the cmex10 option to ensure that
% only type 1 fonts will utilized at all point sizes. Without this option,
% it is possible that some math symbols, particularly those within
% footnotes, will be rendered in bitmap form which will result in a
% document that can not be IEEE Xplore compliant!
%
% Also, note that the amsmath package sets \interdisplaylinepenalty to 10000
% thus preventing page breaks from occurring within multiline equations. Use:
\interdisplaylinepenalty=2500
% after loading amsmath to restore such page breaks as IEEEtran.cls normally
% does. amsmath.sty is already installed on most LaTeX systems. The latest
% version and documentation can be obtained at:
% http://www.ctan.org/tex-archive/macros/latex/required/amslatex/math/





% *** SPECIALIZED LIST PACKAGES ***
%
\usepackage{algorithmic}
% algorithmic.sty was written by Peter Williams and Rogerio Brito.
% This package provides an algorithmic environment fo describing algorithms.
% You can use the algorithmic environment in-text or within a figure
% environment to provide for a floating algorithm. Do NOT use the algorithm
% floating environment provided by algorithm.sty (by the same authors) or
% algorithm2e.sty (by Christophe Fiorio) as IEEE does not use dedicated
% algorithm float types and packages that provide these will not provide
% correct IEEE style captions. The latest version and documentation of
% algorithmic.sty can be obtained at:
% http://www.ctan.org/tex-archive/macros/latex/contrib/algorithms/
% There is also a support site at:
% http://algorithms.berlios.de/index.html
% Also of interest may be the (relatively newer and more customizable)
% algorithmicx.sty package by Szasz Janos:
% http://www.ctan.org/tex-archive/macros/latex/contrib/algorithmicx/




% *** ALIGNMENT PACKAGES ***
%
%\usepackage{array}
% Frank Mittelbach's and David Carlisle's array.sty patches and improves
% the standard LaTeX2e array and tabular environments to provide better
% appearance and additional user controls. As the default LaTeX2e table
% generation code is lacking to the point of almost being broken with
% respect to the quality of the end results, all users are strongly
% advised to use an enhanced (at the very least that provided by array.sty)
% set of table tools. array.sty is already installed on most systems. The
% latest version and documentation can be obtained at:
% http://www.ctan.org/tex-archive/macros/latex/required/tools/


%\usepackage{mdwmath}
%\usepackage{mdwtab}
% Also highly recommended is Mark Wooding's extremely powerful MDW tools,
% especially mdwmath.sty and mdwtab.sty which are used to format equations
% and tables, respectively. The MDWtools set is already installed on most
% LaTeX systems. The lastest version and documentation is available at:
% http://www.ctan.org/tex-archive/macros/latex/contrib/mdwtools/


% IEEEtran contains the IEEEeqnarray family of commands that can be used to
% generate multiline equations as well as matrices, tables, etc., of high
% quality.


%\usepackage{eqparbox}
% Also of notable interest is Scott Pakin's eqparbox package for creating
% (automatically sized) equal width boxes - aka "natural width parboxes".
% Available at:
% http://www.ctan.org/tex-archive/macros/latex/contrib/eqparbox/





% *** SUBFIGURE PACKAGES ***
%\usepackage[tight,footnotesize]{subfigure}
% subfigure.sty was written by Steven Douglas Cochran. This package makes it
% easy to put subfigures in your figures. e.g., "Figure 1a and 1b". For IEEE
% work, it is a good idea to load it with the tight package option to reduce
% the amount of white space around the subfigures. subfigure.sty is already
% installed on most LaTeX systems. The latest version and documentation can
% be obtained at:
% http://www.ctan.org/tex-archive/obsolete/macros/latex/contrib/subfigure/
% subfigure.sty has been superceeded by subfig.sty.



%\usepackage[caption=false]{caption}
%\usepackage[font=footnotesize]{subfig}
% subfig.sty, also written by Steven Douglas Cochran, is the modern
% replacement for subfigure.sty. However, subfig.sty requires and
% automatically loads Axel Sommerfeldt's caption.sty which will override
% IEEEtran.cls handling of captions and this will result in nonIEEE style
% figure/table captions. To prevent this problem, be sure and preload
% caption.sty with its "caption=false" package option. This is will preserve
% IEEEtran.cls handing of captions. Version 1.3 (2005/06/28) and later 
% (recommended due to many improvements over 1.2) of subfig.sty supports
% the caption=false option directly:
%\usepackage[caption=false,font=footnotesize]{subfig}
%
% The latest version and documentation can be obtained at:
% http://www.ctan.org/tex-archive/macros/latex/contrib/subfig/
% The latest version and documentation of caption.sty can be obtained at:
% http://www.ctan.org/tex-archive/macros/latex/contrib/caption/




% *** FLOAT PACKAGES ***
%
%\usepackage{fixltx2e}
% fixltx2e, the successor to the earlier fix2col.sty, was written by
% Frank Mittelbach and David Carlisle. This package corrects a few problems
% in the LaTeX2e kernel, the most notable of which is that in current
% LaTeX2e releases, the ordering of single and double column floats is not
% guaranteed to be preserved. Thus, an unpatched LaTeX2e can allow a
% single column figure to be placed prior to an earlier double column
% figure. The latest version and documentation can be found at:
% http://www.ctan.org/tex-archive/macros/latex/base/



%\usepackage{stfloats}
% stfloats.sty was written by Sigitas Tolusis. This package gives LaTeX2e
% the ability to do double column floats at the bottom of the page as well
% as the top. (e.g., "\begin{figure*}[!b]" is not normally possible in
% LaTeX2e). It also provides a command:
%\fnbelowfloat
% to enable the placement of footnotes below bottom floats (the standard
% LaTeX2e kernel puts them above bottom floats). This is an invasive package
% which rewrites many portions of the LaTeX2e float routines. It may not work
% with other packages that modify the LaTeX2e float routines. The latest
% version and documentation can be obtained at:
% http://www.ctan.org/tex-archive/macros/latex/contrib/sttools/
% Documentation is contained in the stfloats.sty comments as well as in the
% presfull.pdf file. Do not use the stfloats baselinefloat ability as IEEE
% does not allow \baselineskip to stretch. Authors submitting work to the
% IEEE should note that IEEE rarely uses double column equations and
% that authors should try to avoid such use. Do not be tempted to use the
% cuted.sty or midfloat.sty packages (also by Sigitas Tolusis) as IEEE does
% not format its papers in such ways.


%\ifCLASSOPTIONcaptionsoff
%  \usepackage[nomarkers]{endfloat}
% \let\MYoriglatexcaption\caption
% \renewcommand{\caption}[2][\relax]{\MYoriglatexcaption[#2]{#2}}
%\fi
% endfloat.sty was written by James Darrell McCauley and Jeff Goldberg.
% This package may be useful when used in conjunction with IEEEtran.cls'
% captionsoff option. Some IEEE journals/societies require that submissions
% have lists of figures/tables at the end of the paper and that
% figures/tables without any captions are placed on a page by themselves at
% the end of the document. If needed, the draftcls IEEEtran class option or
% \CLASSINPUTbaselinestretch interface can be used to increase the line
% spacing as well. Be sure and use the nomarkers option of endfloat to
% prevent endfloat from "marking" where the figures would have been placed
% in the text. The two hack lines of code above are a slight modification of
% that suggested by in the endfloat docs (section 8.3.1) to ensure that
% the full captions always appear in the list of figures/tables - even if
% the user used the short optional argument of \caption[]{}.
% IEEE papers do not typically make use of \caption[]'s optional argument,
% so this should not be an issue. A similar trick can be used to disable
% captions of packages such as subfig.sty that lack options to turn off
% the subcaptions:
% For subfig.sty:
% \let\MYorigsubfloat\subfloat
% \renewcommand{\subfloat}[2][\relax]{\MYorigsubfloat[]{#2}}
% For subfigure.sty:
% \let\MYorigsubfigure\subfigure
% \renewcommand{\subfigure}[2][\relax]{\MYorigsubfigure[]{#2}}
% However, the above trick will not work if both optional arguments of
% the \subfloat/subfig command are used. Furthermore, there needs to be a
% description of each subfigure *somewhere* and endfloat does not add
% subfigure captions to its list of figures. Thus, the best approach is to
% avoid the use of subfigure captions (many IEEE journals avoid them anyway)
% and instead reference/explain all the subfigures within the main caption.
% The latest version of endfloat.sty and its documentation can obtained at:
% http://www.ctan.org/tex-archive/macros/latex/contrib/endfloat/
%
% The IEEEtran \ifCLASSOPTIONcaptionsoff conditional can also be used
% later in the document, say, to conditionally put the References on a 
% page by themselves.





% *** PDF, URL AND HYPERLINK PACKAGES ***
%
%\usepackage{url}
% url.sty was written by Donald Arseneau. It provides better support for
% handling and breaking URLs. url.sty is already installed on most LaTeX
% systems. The latest version can be obtained at:
% http://www.ctan.org/tex-archive/macros/latex/contrib/misc/
% Read the url.sty source comments for usage information. Basically,
% \url{my_url_here}.





% *** Do not adjust lengths that control margins, column widths, etc. ***
% *** Do not use packages that alter fonts (such as pslatex).         ***
% There should be no need to do such things with IEEEtran.cls V1.6 and later.
% (Unless specifically asked to do so by the journal or conference you plan
% to submit to, of course. )

\begin{document}
%
% paper title
% can use linebreaks \\ within to get better formatting as desired
\title{Generating NFL Game Headlines from Box Score Statistics}
%
%
% author names and IEEE memberships
% note positions of commas and nonbreaking spaces ( ~ ) LaTeX will not break
% a structure at a ~ so this keeps an author's name from being broken across
% two lines.
% use \thanks{} to gain access to the first footnote area
% a separate \thanks must be used for each paragraph as LaTeX2e's \thanks
% was not built to handle multiple paragraphs
%

\author{\{pwgreene, ibelson, jaclarke\}@mit.edu}

% note the % following the last \IEEEmembership and also \thanks - 
% these prevent an unwanted space from occurring between the last author name
% and the end of the author line. i.e., if you had this:
% 
% \author{....lastname \thanks{...} \thanks{...} }
%                     ^------------^------------^----Do not want these spaces!
%
% a space would be appended to the last name and could cause every name on that
% line to be shifted left slightly. This is one of those "LaTeX things". For
% instance, "\textbf{A} \textbf{B}" will typeset as "A B" not "AB". To get
% "AB" then you have to do: "\textbf{A}\textbf{B}"
% \thanks is no different in this regard, so shield the last } of each \thanks
% that ends a line with a % and do not let a space in before the next \thanks.
% Spaces after \IEEEmembership other than the last one are OK (and needed) as
% you are supposed to have spaces between the names. For what it is worth,
% this is a minor point as most people would not even notice if the said evil
% space somehow managed to creep in.



% The paper headers
%\markboth{Journal of \LaTeX\ Class Files,~Vol.~6, No.~1, January~2007}%
%{Shell \MakeLowercase{\textit{et al.}}: Bare Demo of IEEEtran.cls for Journals}
% The only time the second header will appear is for the odd numbered pages
% after the title page when using the twoside option.
% 
% *** Note that you probably will NOT want to include the author's ***
% *** name in the headers of peer review papers.                   ***
% You can use \ifCLASSOPTIONpeerreview for conditional compilation here if
% you desire.




% If you want to put a publisher's ID mark on the page you can do it like
% this:
%\IEEEpubid{0000--0000/00\$00.00~\copyright~2007 IEEE}
% Remember, if you use this you must call \IEEEpubidadjcol in the second
% column for its text to clear the IEEEpubid mark.



% use for special paper notices
%\IEEEspecialpapernotice{(Invited Paper)}




% make the title area
\maketitle

% IEEEtran.cls defaults to using nonbold math in the Abstract.
% This preserves the distinction between vectors and scalars. However,
% if the journal you are submitting to favors bold math in the abstract,
% then you can use LaTeX's standard command \boldmath at the very start
% of the abstract to achieve this. Many IEEE journals frown on math
% in the abstract anyway.

% Note that keywords are not normally used for peerreview papers.
% \begin{IEEEkeywords}
% IEEEtran, journal, \LaTeX, paper, template.
% \end{IEEEkeywords}






% For peer review papers, you can put extra information on the cover
% page as needed:
% \ifCLASSOPTIONpeerreview
% \begin{center} \bfseries EDICS Category: 3-BBND \end{center}
% \fi
%
% For peerreview papers, this IEEEtran command inserts a page break and
% creates the second title. It will be ignored for other modes.
\IEEEpeerreviewmaketitle



\section{Introduction}
With the large number of users and diverse use of applications within the MIT wireless network, maintaining performance becomes critical. When devices attempt to connect to the MIT network, it is common for multiple Access Points to be within range so devices will choose to connect to the strongest signal AP. However, this may not be optimal for performance, since signal strength indicates nothing about traffic congestion. Depending on the application, it could be better to connect to an AP whose signal strength is weaker, but has less congestion and thus higher throughput.

In looking to address this issue and optimize performance, MIT facilities seeks for its wireless internet users to be able to connect to Access Points (APs) with acceptable performance, either by connecting to an AP within range or by connecting to a nearby recommended AP, all while still maintaining maximum network utilization across the whole MIT network. In addition, for the purposes of network management, MIT facilities needs to collect data on each AP's amount of data transfer as well as its approximate number of unique users. To accomplish this, we propose Latch, a simple, reliable, and scalable system for optimizing network performance for users and optimizing network utilization.

% needed in second column of first page if using \IEEEpubid
%\IEEEpubidadjcol

% An example of a floating figure using the graphicx package.
% Note that \label must occur AFTER (or within) \caption.
% For figures, \caption should occur after the \includegraphics.
% Note that IEEEtran v1.7 and later has special internal code that
% is designed to preserve the operation of \label within \caption
% even when the captionsoff option is in effect. However, because
% of issues like this, it may be the safest practice to put all your
% \label just after \caption rather than within \caption{}.
%
% Reminder: the "draftcls" or "draftclsnofoot", not "draft", class
% option should be used if it is desired that the figures are to be
% displayed while in draft mode.
%
\section{System Design}
Latch's basic design consists of three modules: Access Points, client users, and an MIT Information System \& Technology (IS\&T) server, as shown in Figure \ref{comm_diagram}.  Access Points are connected to the server, through which all congestion data is transmitted. The server stores this data as well as the locations for APs in order to relay to each AP congestion information about their neighboring APs. Clients connect to APs and send their application requirements as well as a list of nearby APs with acceptable performance. Using this information and the congestion information from the server, the AP uses an optimization algorithm to determine whether the the client should connect and if not, it recommends another nearby acceptable AP to connect to, sending this information to the client as a connection message.

\begin{figure*}[!t]
\centering
\includegraphics[width=2.5in]{john.png}
\caption{Communications Diagram}
\label{comm_diagram}
\end{figure*}

% Note that IEEE typically puts floats only at the top, even when this
% results in a large percentage of a column being occupied by floats.


% An example of a double column floating figure using two subfigures.
% (The subfig.sty package must be loaded for this to work.)
% The subfigure \label commands are set within each subfloat command, the
% \label for the overall figure must come after \caption.
% \hfil must be used as a separator to get equal spacing.
% The subfigure.sty package works much the same way, except \subfigure is
% used instead of \subfloat.
%
%\begin{figure*}[!t]
%\centerline{\subfloat[Case I]\includegraphics[width=2.5in]{subfigcase1}%
%\label{fig_first_case}}
%\hfil
%\subfloat[Case II]{\includegraphics[width=2.5in]{subfigcase2}%
%\label{fig_second_case}}}
%\caption{Simulation results}
%\label{fig_sim}
%\end{figure*}
%
% Note that often IEEE papers with subfigures do not employ subfigure
% captions (using the optional argument to \subfloat), but instead will
% reference/describe all of them (a), (b), etc., within the main caption.


% An example of a floating table. Note that, for IEEE style tables, the 
% \caption command should come BEFORE the table. Table text will default to
% \footnotesize as IEEE normally uses this smaller font for tables.
% The \label must come after \caption as always.
%
%\begin{table}[!t]
%% increase table row spacing, adjust to taste
%\renewcommand{\arraystretch}{1.3}
% if using array.sty, it might be a good idea to tweak the value of
% \extrarowheight as needed to properly center the text within the cells
%\caption{An Example of a Table}
%\label{table_example}
%\centering
%% Some packages, such as MDW tools, offer better commands for making tables
%% than the plain LaTeX2e tabular which is used here.
%\begin{tabular}{|c||c|}
%\hline
%One & Two\\
%\hline
%Three & Four\\
%\hline
%\end{tabular}
%\end{table}


% Note that IEEE does not put floats in the very first column - or typically
% anywhere on the first page for that matter. Also, in-text middle ("here")
% positioning is not used. Most IEEE journals use top floats exclusively.
% Note that, LaTeX2e, unlike IEEE journals, places footnotes above bottom
% floats. This can be corrected via the \fnbelowfloat command of the
% stfloats package.
\subsection{Server}
The data stored in the IS\&T server will primarily be organized in two SQL tables:
\verb|Congestions| and \verb|Local_AP|.

The \verb|Congestions| table maps the MAC address of every AP in the network to its measure of congestion.
It is defined in Listing \ref{code1}.
 
\begin{listing}[H]
\begin{minted}[breaklines]{sql}
CREATE TABLE Congestion (
  address VARCHAR(12),
  bytes_utilized BIGINT,
  capacity BIGINT,
  congestion BIGINT);
\end{minted}
\caption{Congestion Table}
\label{code1}
\end{listing}
 
The \verb|bytes_utilized| field gives a straightforward measurement of remaining AP capacity, while the congestion column is reserved for a separate measure possibly based on dropped packets or user happiness levels. The \verb|Local_AP| table stores each AP in the network with a list of neighboring APs within 625' of it. It is defined in Listing \ref{code2}.
 
\begin{listing}[H]
\begin{minted}[breaklines]{sql}
CREATE TABLE Local_AP (
  address VARCHAR(12),
  local_APs VARCHAR(12) MULTISET);
\end{minted}
\caption{Local\_AP Table}
\label{code2}
\end{listing} 

When the server receives packets from a certain AP, it updates the \verb|Congestions| table appropriately. Then, the server will read all of the congestion data that needs to be sent to this AP from the table. The process happens sequentially—the server receives data from AP a, then responds to a with the congestion data it needs.
 
For example, say an AP with MAC address \verb|01-23-45-67-89-ab| sends its congestion data as two variables b,c=(\# of bytes utilized, congestion) to the IS\&T server. The IS\&T server will run the following query in Listing \ref{code3} to update the congestion table.
 
\begin{listing}[H]
\begin{minted}[breaklines]{sql}
UPDATE Congestions
  SET bytes_utilized=b,congestion=c
  WHERE address='01-23-45-67-89-ab';
\end{minted}
\caption{Congestion Updates}
\label{code3}
\end{listing}
Then it will retrieve a list of the APs nearby as in Listing \ref{code4}.
 
\begin{listing}[H]
\begin{minted}[breaklines]{sql}
SELECT local_aps
  FROM Local_AP
  WHERE address='01-23-45-67-89-ab';
\end{minted}
\caption{Local AP Query}
\label{code4}
\end{listing}
 
After it has this list of MAC addresses, it will run a query into \verb|Congestions| to gather a list of each AP and its corresponding congestion data. The procedure is defined in pseudo-code in Listing \ref{code5}.
 
\begin{listing}[H]
\begin{minted}[breaklines]{sql}
For MAC_address in local_aps:
  data <- SELECT congestion
          FROM Congestions
          WHERE address=MAC_address
send data to 01-23-45-67-89-ab
\end{minted}
\caption{Congestion Response to AP}
\label{code5}
\end{listing}
 
It will forward this result to the AP that initially sent its data to the server.
 
Because the congestion data is sent to the server every 2 minutes, the response data about local APs that the IS\&T server sends back to the AP will never be more than 2 minutes out of date. Additionally, because the \verb|Local_AP| table will be dealing with many reads in a single second, the reads can be distributed among the server's many cores, so that the process of retrieving the data can be done concurrently and responses will be nearly instantaneous. The update to the \verb|Congestion| table, however, limits the ability for reads to happen concurrently. Therefore, we let the update happen only once every 30 seconds on a single, separate thread with a copy of the outdated table. Once the update is complete, it will replace the outdated table with the new updated \verb|Congestion| table. The server needs to store a buffer that contains all of the congestion data it received since the last update of this table so that it can do these updates with batches of the congestion data.
 
The number of APs in the system at any given time is less than 4000, which means the server will be doing at most 33 reads/second from the \verb|Congestion| and \verb|Local_AP| tables, and every 30 seconds the server will need to update the approximately 500 entries in the \verb|Congestion| table. We want to ensure that the APs are not all sending their data to the server at the same time so as to ensure the maximum number of packets received by the server at any given 30 second interval is unlikely to be too much higher than 500.
\subsection{Initial Connection Processes}
The client to AP connection process begins with a series of steps to find the optimal AP to which a client should be linked.
 
\subsubsection{Client Phase}
When a client attempts to initially connect to an access point, it needs to intelligently determine which APs it should try first. The following sequence creates an ordered list for the client to use to find the optimal AP.
 
For each channel from 1 to 11, the client will switch to that channel and wait for a heartbeat. As this happens, the client stores a list of tuples for the APs that it finds in the form [(AP MAC Address, AP to Client signal strength)]. This search process takes at most 385ms, allowing 5ms to switch channels and 30ms to wait for a heartbeat. The client then filters the list by filter strength, keeping only the APs whose strength it deems acceptable (this measure of acceptability is determined by the client itself, e.g. all signals above 60\%). Next, it sorts the list in descending order by strength. These operations will take a small amount time, as the client is only searching on 11 channels, so there can only be at most 11 APs.
 
Once the client has this list, it will connect to the “best,” or strongest AP. If this AP is too congested, it will make space for the new client using the Kickoff Algorithm described in the next section. The AP will tell the clients who got kicked off which APs they should connect to next.
 
\subsubsection{AP Phase}
When a new client connects to an AP, the AP needs to know if it has room to support the new client. By storing certain relevant information and running a clever congestion analysis algorithm, the AP will know if it has room, and how to make room if it does not.
 
Each AP stores a table mapping its geographically neighboring APs to their levels of congestion. It also holds a list of its currently connected clients and the APs to which they have reasonable signal strength.
 
Every two minutes, the AP contacts the IS\&T server. The AP tells the server its own congestion level, and the server sends back an updated table of the nearby AP congestion levels. A metric for determining the level of congestion is described in a later section.
 
The AP sorts its list of connected clients in order of whom it should kick off first, using the Kickoff Algorithm described in the next section. If the AP's congestion level is too high, it kicks clients off the AP, starting at the top of the list, until it is no longer congested. When it kicks a client off, it tells them which of their acceptable APs is the best one for them to connect to. If the congestion level is acceptable and there is room for the new client, the AP simply allows the connection to continue and includes the new client in its next congestion analysis.

\subsection{Kick-off Algorithm}
Because our APs allow any client to connect, we can end up in a situation where
too many clients have connected. To prevent congestion, we must prune the
client list regularly.

In order to do the pruning, we maintain a list of connected clients, sorted in
the order we would like to kick them off. When a new client connects, we just
insert them into the list at the correct position. When our AP congestion map
changes, we have to re-sort the entire list (but this happens
infrequently).

Then we regularly measure our own congestion, and if it is above
acceptable levels we kick clients from the top of the list until we have some
breathing room.

So the question is how to rank the clients. We choose to rank them in terms of
how good their next best alternative AP is. To do this, we need to know what
APs each client can connect to, how much data the client needs, and how
congested those APs are.

This scheme also has the advantage of calculating the best alternative for
every client--so when we kick that client, we tell them which AP to try to
reconnect to.
    
When a client connects, we get a list from them that tells us which APs they
are happy to connect to. We assume this list never changes until they
disconnect--we never update it. If the client does not move around much, the
signal strengths they have to various APs should not change much, and none of
the APs should cross from unacceptable to acceptable or vice versa. In the
worst case, if the client for example walks around, we might kick them and
redirect them to an AP they can no longer reach. Then they would re-initiate
their scanning routine, and re-connect to us with a new list of acceptable
APs--and this would only take a few seconds.

The client has the information about their signal strengths to various APs, and
the AP knows how congested the nearby APs are. So the score for a client C with
desired data rate A is the congestion on the least congested AP that C could
connect to with the ability to support A more data.

That's quite a mouthful, so Python pseudo-code is shown in Listing \ref{score_client}.
\begin{listing}[H]
\begin{minted}[breaklines]{python}
def score_client(c):
  desired_data_rate = c['data']
  acceptable_aps = [a for a in c['aps'] if free_space(a) > desired_data_rate]
  return max(map(congestion, acceptable_aps))
\end{minted}
\caption{Client Kick-off-ability}
\label{score_client}
\end{listing}

Our APs are memoryless, in the sense that we don't keep track of who has
connected to us in the past. That's a nice property to avoid using tons of
memory, but it means that we might end up in a bad infinite loop if we're not
careful. If we redirect clients to an AP that cannot support them, that AP will
kick them out as well--and if it kicks them back to us, we will just kick the
client right back to the same AP. The loop would continue until either one of
the APs could support the client, or the APs updated their congestion
measurements and redirected differently. The former might not happen, and the
latter takes up to two minutes (and might not even fix the problem). If this
ever occurs, we will have an unhappy client--so measuring our congestion
conservatively is very important. If other APs think we have space but we
don't, the consequences are very bad. If they think we don't have space when we
just barely do, they just won't redirect users to us.

Getting an accurate measure of congestion isn't necessarily easy. In our
design, we calculate the amount of free bandwidth we would have if each client
used their full requested data rate, and call that the amount of free space we
have. We have designed our system to have room for a different measurement of
congestion--such as the number of unhappy users, or some function of the times
we drop packets. For our final report, we intend to analyse the properties of
these various congestion measures and decide on one which makes our algorithm
most robust.

Another problem with this scheme is that we don't redirect any clients until
an AP is full.
We could add some procedure to the AP that would try to prune clients if the
neighboring APs were significantly less congested, but we think the unhappiness
from balancing users who are getting adequate service won't be worth it.
\section{Conclusion}

Latch is a system for optimizing user's network performance while maximizing network utilization. It is designed for simplicity, reliability, as well as scalability. The system also meets the requirements for network management of MIT's network in collecting data on the APs' amount of data transfer and number of users. This is accomplished by leveraging the use of our designed algorithms for constant optimization updates and effective use of data models.

Currently, the algorithms used don't take care of the balancing of clients, until congestion is present. Also, the client's unhappiness is not taken into consideration during the network balancing of clients. Our further work will be geared towards improving the optimization algorithms both to take care of client balancing before any congestion as well as to make usage of clients' happiness in balancing optimization.




% if have a single appendix:
%\appendix[Proof of the Zonklar Equations]
% or
%\appendix  % for no appendix heading
% do not use \section anymore after \appendix, only \section*
% is possibly needed

% use appendices with more than one appendix
% then use \section to start each appendix
% you must declare a \section before using any
% \subsection or using \label (\appendices by itself
% starts a section numbered zero.)
%

% Can use something like this to put references on a page
% by themselves when using endfloat and the captionsoff option.

% trigger a \newpage just before the given reference
% number - used to balance the columns on the last page
% adjust value as needed - may need to be readjusted if
% the document is modified later
%\IEEEtriggeratref{8}
% The "triggered" command can be changed if desired:
%\IEEEtriggercmd{\enlargethispage{-5in}}

% references section

% biography section
% 
% If you have an EPS/PDF photo (graphicx package needed) extra braces are
% needed around the contents of the optional argument to biography to prevent
% the LaTeX parser from getting confused when it sees the complicated
% \includegraphics command within an optional argument. (You could create
% your own custom macro containing the \includegraphics command to make things
% simpler here.)
%\begin{biography}[{\includegraphics[width=1in,height=1.25in,clip,keepaspectratio]{mshell}}]{Michael Shell}
% or if you just want to reserve a space for a photo:


% You can push biographies down or up by placing
% a \vfill before or after them. The appropriate
% use of \vfill depends on what kind of text is
% on the last page and whether or not the columns
% are being equalized.

%\vfill

% Can be used to pull up biographies so that the bottom of the last one
% is flush with the other column.
%\enlargethispage{-5in}



% that's all folks
\end{document}

